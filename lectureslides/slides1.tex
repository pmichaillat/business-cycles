\documentclass[11pt,aspectratio=169,xcolor={dvipsnames},hyperref={pdftex,pdfpagemode=UseNone,hidelinks,pdfdisplaydoctitle=true},usepdftitle=false]{beamer}
\usepackage{presentation,lecture,math}
\hypersetup{pdftitle={Accounting for Business Cycles}}
\newcommand{\npdf}{../figures/figures1.pdf}
\newcommand{\wpdf}{../figures/widefigures1.pdf}
\begin{document}

\title{Accounting for Business Cycles}
\information{Pascal Michaillat}%
{December 2023}%
{https://pascalmichaillat.org/z/}
\frame{\titlepage}

\begin{frame}
\frametitle{Outline}
\begin{itemize}
\item Define business cycles
\item Define capacity and its utilization
\item Show that capacity is acyclical
\item While the utilization of capacity is procyclical 
\begin{itemize}
\item[\then] Economic slack is countercyclical
\end{itemize}
\item Discuss the cost of economic slack and need for stabilization
\end{itemize}	
\end{frame}

\begin{frame}
\heading{Definition of business cycles}
\end{frame}

\begin{frame}
\frametitle{Definition of business cycles (Burns, Mitchell 1946)}
\begin{quote}
A cycle consists of expansions occurring at about the same time in \al{many economic activities}, followed by similarly general recessions, contractions, and revivals which merge into the expansion phase of the next cycle; this sequence of changes is \al{recurrent but not periodic}; in duration business cycles vary from \al{more than one year to ten or twelve years}; they are not divisible into shorter cycles of similar character with amplitudes approximating their own.
\end{quote}	
\end{frame}

\begin{frame}
\frametitle{Industrial production fluctuations (Stock, Watson 1999)}
\includegraphics<1>[scale=\wfig,page=1]{\wpdf}%
\end{frame}

\begin{frame}
\frametitle{Real GDP fluctuations (Stock, Watson 1999)}
\includegraphics<1>[scale=\wfig,page=3]{\wpdf}%
\end{frame}


\begin{frame}
\frametitle{Consumption fluctuations (Stock, Watson 1999)}
\includegraphics<1>[scale=\wfig,page=4]{\wpdf}%
\end{frame}

\begin{frame}
\heading{Business-cycle accounting}
\end{frame}

\begin{frame}
\frametitle{Simple accounting framework}
\begin{itemize}	
\item Production identity:
\begin{equation*}
\text{production} = \text{productive capacity} \times \text{utilization}
\end{equation*}
\item Expression in terms of $\text{slack} = 1 - \text{utilization}$
\begin{equation*}
\text{production} = \text{productive capacity} \times [1 - \text{slack}]
\end{equation*}
\item Possible source of business-cycle fluctuations:
\begin{itemize}
	\item Fluctuations in productive capacity
	\item Fluctuations in slack
	\item Or both
\end{itemize}
\end{itemize}	
\end{frame}

\begin{frame}
\frametitle{Source of business-cycle fluctuations in the literature}
\begin{itemize}
\item General Disequilibrium model (1970s): fluctuations in slack
\begin{itemize}
	\item Markets are not clearing so slack $> 0$
	\item Fluctuations in slack due to aggregate demand shocks
\end{itemize}
\item Real Business-Cycle model (1980s–1990s): fluctuations in capacity
\begin{itemize}
	\item Markets are competitive so utilization $= 100\%$
	\item Fluctuations in capacity due to technology shocks
\end{itemize}
\item New Keynesian model (2000s–2010s): fluctuations in capacity
\begin{itemize}
	\item Markets are monopolistically competitive so utilization $= 100\%$
	\item Fluctuations in capacity due to aggregate demand shocks and response of participation
\end{itemize}
\end{itemize}
\end{frame}

\begin{frame}
\heading{Capacity is acyclical}
\end{frame}

\begin{frame}
\frametitle{Inputs into productive capacity}
\begin{itemize}
\item Capital stock
\item Labor force: share of the working-age civilian population that wishes to work and is available
\item Technology
\item All inputs are combined through production process (production function) 
\item[\so] Productive capacity
\end{itemize}
\end{frame}

\begin{frame}
\frametitle{Capital stock in constant dollars}
\includegraphics<1>[scale=\wfig,page=7]{\wpdf}%
\end{frame}

\begin{frame}
\frametitle{Civilian population}
\includegraphics<1>[scale=\wfig,page=8]{\wpdf}%
\end{frame}

\begin{frame}
\frametitle{Civilian labor force}
\includegraphics<1>[scale=\wfig,page=9]{\wpdf}%
\end{frame}

\begin{frame}
\frametitle{Labor-force participation rate (prime age)}
\includegraphics<1>[scale=\wfig,page=10]{\wpdf}%
\end{frame}

\begin{frame}
\frametitle{Labor-force participation rate (men)}
\includegraphics<1>[scale=\wfig,page=11]{\wpdf}%
\end{frame}

\begin{frame}
\frametitle{Labor-force participation rate (women)}
\includegraphics<1>[scale=\wfig,page=12]{\wpdf}%
\end{frame}

\begin{frame}
\frametitle{Proxies for technology}
\includegraphics<1>[scale=\wfig,page=13]{\wpdf}%
\includegraphics<2>[scale=\wfig,page=14]{\wpdf}%
\end{frame}

\begin{frame}
\frametitle{Technology: difficult to measure but likely acyclical}
\begin{itemize}
\item In practice, technology is bound to be unrelated to business cycles
\begin{itemize}
	\item Invention process is slow and random
	\item Diffusion process is slow and random
	\item Depreciation process (loss of know-how) is slow and random
\end{itemize}
\item Total factor productivity (TFP) is procyclical
\begin{itemize}
	\item But TFP is a residual, driven almost exclusively by factor utilization
\end{itemize}
\end{itemize}
\end{frame}

\begin{frame}
\frametitle{Capacity utilization \so TFP (Stock, Watson 1999)}
\includegraphics<1>[scale=\wfig,page=15]{\wpdf}%
\end{frame}

\begin{frame}
\heading{Slack is countercyclical}
\end{frame}

\begin{frame}
\frametitle{Occurrences of slack}
\begin{itemize}
	\item Slack on the labor market:
	\begin{equation*}
	\text{employment} = (1 - \text{unemployment}) \times \text{labor force}
	\end{equation*}
	\item Slack on the product market with production function $f$:
	\begin{equation*}
	\text{production} = (1 - \text{idleness}) \times f(\text{technology}, \text{capital}, \text{employment})
	\end{equation*}
	\item Two forms of slack
	\begin{itemize}
		\item Unemployment: share of labor force that is not employed
		\item Idleness: share of firms' capacity (based on current capital stock and number of employees) that is not in use 
	\end{itemize}
\end{itemize}
\end{frame}

\begin{frame}
\frametitle{Unemployment rate (Stock, Watson 1999)}
\includegraphics<1>[scale=\wfig,page=16]{\wpdf}%
\end{frame}

\begin{frame}
\frametitle{Unemployment rate (\href{https://pascalmichaillat.org/13/}{Michaillat, Saez 2022})}
\includegraphics[scale=\wfig,page=17]{\wpdf}%
\end{frame}

\begin{frame}
\frametitle{Other unemployment: 1\% of labor force in grey area}
\begin{columns}
\begin{column}{0.6\textwidth}
\includegraphics[scale=\sfig,page=1]{\npdf}%
\end{column}
\begin{column}{0.4\textwidth}
\begin{itemize}
	\item U3 = people wanting to work, available to work, and having searched for a job in past 4 weeks
	\item U4 = U3 + people people discouraged by lack of jobs
	\item U5 = U4 + people having searched for job in past year
\end{itemize}
\end{column}  
\end{columns}                    
\end{frame}

\begin{frame}
\frametitle{Operating rate and idleness rate}
\begin{itemize}
\item Data from \href{https://www.ismworld.org/}{Institute for Supply Management} (ISM)
\item Semi-annual Economic Forecast: \href{https://www.ismworld.org/supply-management-news-and-reports/reports/semi-annual-economic-forecast/2022/fall/}{Fall 2022 example}
\item Operating rate = actual production level of firms as a share of their maximum production level given current capital and labor (normal capacity)
\item (Note that the ISM reports a lot of other interesting statistics on slack and tightness)
\end{itemize}	
\end{frame}

\begin{frame}
\frametitle{Idleness rate in manufacturing (ISM survey)}
\includegraphics<1>[scale=\wfig,page=18]{\wpdf}%
\end{frame}

\begin{frame}
\frametitle{Idleness rate in services (ISM survey)}
\includegraphics<1>[scale=\wfig,page=19]{\wpdf}%
\end{frame}

\begin{frame}
\heading{Fluctuations in slack account for business cycles}
\end{frame}

\begin{frame}
\frametitle{Fluctuations in GPD and consumption}
\includegraphics<1>[scale=\wfig,page=20]{\wpdf}%
\includegraphics<2>[scale=\wfig,page=21]{\wpdf}%
\end{frame}

\begin{frame}
\frametitle{Fluctuations in slack and GDP}
\includegraphics<1>[scale=\wfig,page=22]{\wpdf}%
\includegraphics<2>[scale=\wfig,page=23]{\wpdf}%
\includegraphics<3>[scale=\wfig,page=24]{\wpdf}%
\end{frame}

\begin{frame}
\frametitle{Recasting the results as Okun's laws}
\includegraphics<1>[scale=\nfig,page=2]{\npdf}%
\includegraphics<2>[scale=\nfig,page=3]{\npdf}%
\includegraphics<3>[scale=\nfig,page=4]{\npdf}%
\end{frame}

\begin{frame}
\frametitle{Okun’s law in the United States, 1948–2013 (Ball, Leigh, Loungani 2017)}
\includegraphics<1>[scale=\wfig,page=25]{\wpdf}%
\end{frame}


\begin{frame}
\heading{Cost of slack and need for stabilization}
\end{frame}

\begin{frame}
\frametitle{Non-monetary cost from unemployment}
\begin{itemize}
\item Controlling for income and other personal characteristics, unemployment imposes large costs
\item Di Tella, MacCulloch, Oswald (2003): ``Recessions create psychic losses that extend beyond the fall in GDP and rise in the number of people unemployed. These losses are large.''
\begin{itemize}
	\item US General Social Survey: becoming unemployed is as painful as divorcing or dropping from the top to the bottom income quartile
	\item Euro-barometer survey: becoming unemployed is as bad as losing \$3,500 of income a year, about 45\% of average per-capita income
\end{itemize}
\item Blanchflower, Oswald (2004): becoming unemployed is as bad as losing \$60,000 of income a year, 3 times average yearly per-capita income
\end{itemize}	
\end{frame}

\begin{frame}
\frametitle{Where do the psychological costs of unemployment come from?}
\begin{itemize}
\item Depression, anxiety, and strained personal relations (Eisenberg, Lazarfeld 1938)
\item Job loss is a traumatic event that reduces self-esteem (Akerlof, Yellen 1985)
\item Joblessness reduces psychological well-being by creating a feeling that life is not under one's control (Goldsmith, Darity 1992)
\item The benefits of work that are lost in unemployment include (Jahoda 1981):
\begin{itemize}
	\item Time structure on the working day
	\item Regularly shared experiences and contacts with people outside the nuclear family
	\item Goals and purposes that transcend their own
	\item Source of personal status and identity
	\item Regular activity
\end{itemize}
\end{itemize}	
\end{frame}

\begin{frame}
\frametitle{Social value of nonwork from revealed preferences}
\begin{itemize}
\item Borgschulte, Martorell (2018): natural experiment using military administrative data
\begin{itemize}
\item 420,000 veterans
\item must choose between reenlisting with various bonuses and moving back home to enter local labor market with various unemployment rates \then can compare how higher unemployment rate affects decisions
\item nonwork time has low value: home production + recreation = $13\%$--$35\%$ earnings
\end{itemize}
\item Mas, Pallais (2019): field experiment in which job applicants choose wage-hour bundles
\begin{itemize}
\item 900 subjects
\item home production + recreation = $58\%$ earnings
\item subjects are already partially employed \then upper bound on value of nonwork
\end{itemize}
\end{itemize}
\end{frame}

\begin{frame}
\frametitle{Cost of business cycles and need for stabilization}
\begin{itemize}
\item Slumps lead to elevated slack \then productive capacity is not used \then wasted consumption and welfare
\item Furthermore, slack in the form of unemployment imposes large, additional social costs
\item However, some slack is required to maximize welfare \then because excessive tightness also has costs
\begin{itemize}
\item Time and effort spent recruiting and hiring
\item Time and effort spent shopping and buying
\end{itemize}
\item Slack should be stabilized to a low, adequate level
\end{itemize}	
\end{frame}

\begin{frame}
\frametitle{But current US stabilization remains incomplete (Michaillat, Saez 2022)}
\includegraphics<1>[scale=\wfig,page=26]{\wpdf}%
\end{frame}

\end{document}