\documentclass[11pt,aspectratio=169,xcolor={dvipsnames},hyperref={pdftex,pdfpagemode=UseNone,hidelinks,pdfdisplaydoctitle=true},usepdftitle=false]{beamer}
\usepackage{presentation,lecture,math}
\hypersetup{pdftitle={Taming Business Cycles with Monetary and Fiscal Policy}}
\newcommand{\npdf}{../figures/figures4.pdf}
\begin{document}

\title{Taming Business Cycles with Monetary and Fiscal Policy}
\information{Pascal Michaillat}%
{December 2023}%
{https://pascalmichaillat.org/z/}
\frame{\titlepage}

\begin{frame}
\frametitle{Outline}
\begin{itemize}
\item Develop Beveridgean framework to think about productive efficiency
\begin{itemize}
	\item Compute efficient labor market tightness
	\item Compute efficient unemployment rate
	\item Based on \href{https://pascalmichaillat.org/9/}{Michaillat, Saez (2021)}
\end{itemize}
\item Derive formula for optimal monetary policy
\begin{itemize}
\item Based on \href{https://pascalmichaillat.org/7/}{Michaillat, Saez (2022)}
\end{itemize}
\item Derive formula for optimal government spending
\begin{itemize}
\item Based on \href{https://pascalmichaillat.org/6/}{Michaillat, Saez (2019)}
 \end{itemize}
\end{itemize}
\end{frame}

\begin{frame}
\heading{Beveridgean framework for productive efficiency}
\end{frame}

\begin{frame}
\frametitle{Composition of labor force}
\begin{itemize}
\item Share $u$ of labor force is unemployed
\begin{itemize}
	\item Home production is fraction $\z\in(0,1)$ for market production
\end{itemize}
\item Share $\k\cdot v$ of labor force is employed recruiting
\begin{itemize}
	\item $\k$ recruiter per vacancy
\end{itemize}
\item Share $1-u-\k v$ of labor force is employed producing
\item Social welfare is determined by home production + market production:
\begin{equation*}
SW \propto 1 - u - \k \cdot v + \z \cdot u = 1 - \k \cdot v - (1-\z) \cdot u 
\end{equation*}
\end{itemize}
\end{frame}


\begin{frame}
\frametitle{Beveridgean model of the economy}
\begin{itemize}
\item Maximize social welfare $\Leftrightarrow$ minimize $\k v + (1-\z) u$
\begin{itemize}
\item Special case with $\k=1$ and $\z=0$: minimize $u+v$ (\href{https://pascalmichaillat.org/13/}{Michaillat, Saez (2023)})
\end{itemize}
\item Of course, cannot set $u = v = 0$
\item Beveridge curve: $v(u)$
\begin{itemize}
\item $v$: vacancy rate
\item $u$: unemployment rate
\item $v(u)$: decreasing in $u$, convex
\end{itemize}
\end{itemize}
\end{frame}

\begin{frame}
\frametitle{US Beveridge curve}
\includegraphics<1>[scale=\nfig,page=1]{\npdf}%
\includegraphics<2>[scale=\nfig,page=2]{\npdf}%
\includegraphics<3>[scale=\nfig,page=3]{\npdf}%
\end{frame}

\begin{frame}
\frametitle{Condition for labor-market efficiency}
\includegraphics<1>[scale=\nfig,page=4]{\npdf}%
\includegraphics<2>[scale=\nfig,page=5]{\npdf}%
\includegraphics<3>[scale=\nfig,page=6]{\npdf}%
\includegraphics<4>[scale=\nfig,page=7]{\npdf}%
\end{frame}

\begin{frame}
\frametitle{Tightness gap and unemployment gap}
\includegraphics<1>[scale=\nfig,page=8]{\npdf}%
\includegraphics<2>[scale=\nfig,page=9]{\npdf}%
\includegraphics<3>[scale=\nfig,page=10]{\npdf}%
\includegraphics<4>[scale=\nfig,page=11]{\npdf}%
\includegraphics<5>[scale=\nfig,page=12]{\npdf}%
\end{frame}

\begin{frame}
\frametitle{Graphical characterization of efficiency}
\begin{itemize}
\item Efficiency at tangency point: $v'(u) = MRS_{uv}$
\item Computing the social marginal rate of substitution:
\begin{equation*}
MRS_{uv} = -\frac{\pdx{SW}{u}}{\pdx{SW}{v}} = - \frac{1-\z}{\k}
\end{equation*}
\item Efficiency condition:
\begin{equation*}
v'(u)  = - \frac{1 - \z}{\k} 
\end{equation*}
\end{itemize}
\end{frame}

\begin{frame}
\frametitle{Analytical characterization of efficiency}
\begin{itemize}
\item Efficiency $\Leftrightarrow$ minimize $\k v(u) + (1-\z) u$
\item First-order condition is necessary and sufficient for this convex problem:
\begin{equation*}
\k v'(u) + (1-\z) =0 
\end{equation*}
\item Efficiency condition:
\begin{equation*}
v'(u)  = - \frac{1 - \z}{\k} 
\end{equation*}
\end{itemize}
\end{frame}

\begin{frame}
\frametitle{Sufficient-statistic formula for efficient tightness}
\begin{itemize}
\item Labor market tightness: $\t = v/u$	
\item Beveridge elasticity: 
\begin{equation*}
\e = -\oe{v}{u} = - \frac{u}{v}\cdot\od{v}{u} = - \frac{v'(u)}{\t}>0
\end{equation*}
\item Condition for efficiency:
\begin{align*}
v'(u) &= -\frac{1 - \z}{\k}\\
-\frac{v'(u)}{\t} \cdot\t & =  \frac{1 - \z}{\k}\\
\t  &=  \frac{1 - \z}{\k \cdot \e}
\end{align*}
\end{itemize}
\end{frame}

\begin{frame}
\frametitle{Efficient tightness}
\begin{itemize}
\item Formula in sufficient statistics (valid in any Beveridgean model):
\begin{equation*}
\al{\t^* = \frac{1 - \z}{\k \cdot \e}}
\end{equation*}
\item In the US, in aggregate, $\z \approx 0$, $\k\approx 1$, and $\e\approx 1$ so $\t^*\approx 1$ (\href{https://pascalmichaillat.org/13/}{Michaillat, Saez 2023})
\begin{itemize}
\item $\e$: Beveridge elasticity
\item $\k$: recruiting cost
\item $\z$: social value of nonwork (does not include benefits and transfers)
\end{itemize}
\item But these statistics might take different values in other countries or in specific industries 
\end{itemize}	
\end{frame}

\begin{frame}
\frametitle{Sufficient-statistic formula for efficient unemployment rate}
\begin{itemize}
\item With isoelastic Beveridge curve:
\begin{align*}
v &= A \cdot u^{-\e} \\
\t &= \frac{v}{u} = A \cdot u^{-(\e+1)}\\
u &= (\t/A)^{-1/(\e+1)}\\
u^* &= (\t^*/A)^{-1/(\e+1)}
\end{align*}
\item $u^*$ obtained from $\t^*$ through Beveridge curve:
\begin{equation*}
\frac{u}{u^*} = \bp{\frac{\t}{\t^*}}^{-1/(1+\e)}
\end{equation*}
\end{itemize}
\end{frame}

\begin{frame}
\frametitle{Efficient unemployment rate}
\begin{itemize}
\item Reshuffling the terms in the previous expression gives the efficient unemployment rate:
\begin{equation*}
\al{u^* = \bp{\frac{\k \cdot \e}{1-\z}\cdot v\cdot u^{\e}}^{1/(1+\e)}}
\end{equation*}
\item In the US, in aggregate, $\z \approx 0$, $\k\approx 1$, and $\e\approx 1$ so $u^*\approx \sqrt{uv}$ (\href{https://pascalmichaillat.org/13/}{Michaillat, Saez 2023})
\item Taking logs in the previous expression, we can also link log unemployment and log tightness gaps, which is useful to switch between unemployment and tightness:
\begin{equation*}
\log(u) - \log(u^*) = -\frac{1}{1+\e} \cdot [\log(\t) - \log(\t^*)]
\end{equation*}
\end{itemize}	
\end{frame}

\begin{frame}
\frametitle{Matching models are Beveridgean models}
\begin{itemize}
\item Unemployment is a function of tightness when flows are balanced:
\begin{equation*}
u(\t) = \frac{\l}{\l + f(\t)}
\end{equation*}
\item We can express relationship as a Beveridge curve:
\begin{align*}
u & = \frac{\l}{\l + \o\cdot \t^{1-\h}}\\
\l & = \l \cdot u+ \o\cdot \frac{v^{1-\h}}{u^{1-\h}}\cdot u\\
\l \cdot(1-u) & = \o\cdot v^{1-\h} \cdot u^{\h}
\end{align*}
\item This yields the Beveridge curve---a negative relationship between $v$ and $u$:
\begin{equation*}
\al{v(u) = \bs{\frac{\l \cdot (1-u)}{\o \cdot u^{\h}}}^{1/(1-\h)}}
\end{equation*}
\end{itemize}
\end{frame}

\begin{frame}
\frametitle{Beveridge elasticity in matching models}
\begin{itemize}
\item Beveridge elasticity is elasticity of vacancy with respect to unemployment along the Beveridge curve:
\begin{align*}
\e & = -\oe{v}{u} = -\frac{1}{1-\h}\cdot \bs{\oe{\l \cdot (1-u)}{u}- \h}\\
\e &= \frac{1}{1-\h}\cdot \bs{ \h - \oe{1-u}{u}}\\
\e &= \frac{1}{1-\h}\bs{\h+ \frac{u}{1-u}}
\end{align*}
\item Since $u/(1-u)$ is small, because $u$ is small, $\e$ is almost constant:
\begin{equation*}
\e \approx \frac{\h}{1-\h}
\end{equation*}
\item $\e \approx 1$ when $\h = 0.5$, which is a standard calibration (Petrongolo, Pissarides 2001)
\end{itemize}
\end{frame}

\begin{frame}
\heading{Optimal monetary policy in dynamic business-cycle model}
\end{frame}

\begin{frame}
\frametitle{Response to excessive tightness}
\includegraphics<1>[scale=\nfig,page=13]{\npdf}%
\includegraphics<2>[scale=\nfig,page=14]{\npdf}%
\end{frame}


\begin{frame}
\frametitle{Response to insufficient tightness}
\includegraphics<1>[scale=\nfig,page=15]{\npdf}%
\includegraphics<2>[scale=\nfig,page=16]{\npdf}%
\includegraphics<3>[scale=\nfig,page=17]{\npdf}%
\end{frame}


\begin{frame}
\frametitle{ZLB constraint}
\includegraphics<1>[scale=\nfig,page=18]{\npdf}%
\includegraphics<2>[scale=\nfig,page=19]{\npdf}%
\end{frame}

\begin{frame}
\frametitle{Wealth tax undoes ZLB}
\includegraphics<1>[scale=\nfig,page=20]{\npdf}%
\end{frame}

\begin{frame}
\heading{Sufficient-statistic formula for optimal monetary policy}
\end{frame}

\begin{frame}
\frametitle{Optimal monetary policy formula}
\begin{itemize}
\item Unemployment rate is function $u(i)$ of interest rate	
\item Linear expansion of $u(i)$ around suboptimal $\bs{i,u}$, assessed at efficient $\bs{i^*,u^*}$:
\begin{equation*}
u^* \approx u + \od{u}{i} \cdot (i^* - i)
\end{equation*}
\item Reshuffling terms yields sufficient-statistic formula:
\begin{equation*}
i - i^* \approx  \frac{u-u^*}{du/di}
\end{equation*}
\item Two sufficient statistics required:
\begin{itemize}
\item Unemployment gap: $u-u^*$
\item Monetary multiplier: $du/di$
\end{itemize}
\end{itemize}
\end{frame}

\begin{frame}
\frametitle{Monetary multiplier in the US:  $du/di \approx 0.5$}
\begin{table}
\begin{tabular*}{\textwidth}{@{\extracolsep\fill}lcc}
Study &   $du/di$ & Method \\
\toprule
Bernanke, Blinder (1992) & $0.6$ & VAR  \\
Leeper, Sims, Zha (1996) & $0.1$ & VAR \\
Christiano, Eichenbaum, Evans (1996) &  $0.1$ & VAR \\
Romer, Romer (2003)  &  $0.9$ & narrative \\
Bernanke, Boivin, Eliasz (2005) &  $0.2$ & FAVAR\\
Coibion (2012)  &  $0.5$ & narrative \& VAR \\
\midrule	
Median  &  $0.5$ &  \\
\bottomrule
\end{tabular*}
\end{table}
\end{frame}

\begin{frame}
\frametitle{Practical rule for monetary policy}
\begin{itemize}
\item Using US evidence on the monetary multiplier, optimal monetary policy becomes:
\begin{equation*}
i - i^* \approx  \frac{u-u^*}{0.5} = 2 \times (u-u^*)
\end{equation*}
\item[\then] Fed should reduce interest rate by 2 percentage points for each positive percentage point of unemployment gap
\item[\then] Fed should raise interest rate by 2 percentage points for each negative percentage point of unemployment gap
\end{itemize}	
\end{frame}

\begin{frame}
\frametitle{Response of Fed to unemployment rate (Bernanke, Blinder 1992)}
\begin{columns}
\begin{column}{0.55\textwidth}
\includegraphics[scale=\nfig,page=21]{\npdf}%
\end{column}
\begin{column}{0.45\textwidth}
\begin{itemize}
	\item Fed funds rate (FFR) drops by 0.28pp when unemployment increases by 0.18pp
	\item Since $u^*$ is very stable, FFR drops by 0.28pp when unemployment gap increases by $\approx 0.18$pp
	\item FFR drops by $0.28/0.18 = 1.6$pp when unemployment gap increases by 1pp
	\item Close to the 2pp response suggested by optimal formula
\end{itemize}
\end{column}  
\end{columns}
\end{frame}

\begin{frame}
\frametitle{Response of Fed during pandemic (\href{https://pascalmichaillat.org/13/}{Michaillat, Saez 2023})}
\begin{columns}
\begin{column}{0.55\textwidth}
\includegraphics[scale=\nfig,page=22]{\npdf}%
\end{column}
\begin{column}{0.45\textwidth}
\begin{itemize}
	\item FFR should drops by $6.3 \times 2 = 12.6$pp at peak of recessions \then ZLB
	\item FFR should have started to increase in 2021Q2, when unemployment gap turned negative
	\item FFR increased by $5.25$pp, so we can expect unemployment to increase by $5.25\times 0.5 = 2.6$pp \then unemployment gap likely to turn positive
	\item Lag of 1–1.5 years for full effect
\end{itemize}
\end{column}  
\end{columns}
\end{frame}

\begin{frame}
\heading{Sufficient-statistic formula for optimal government spending}
\end{frame}

\begin{frame}
\frametitle{Government's problem}
\begin{itemize}
\item Households' flow utility over public and private employment: $\Uc(c,g)$
\item To simplify: set up from the paper on $u^* = \sqrt{uv}$
\begin{itemize}
\item No home production, one recruiter per vacancy 
\end{itemize}
\item Public expenditure is financed by a lump-sum tax to maintain a balanced budget
\item Private producers: $c = 1 - u - v - g$
\item First constraint: Beveridge curve $v(u)$
\item Second constraint: public spending affects unemployment $u(g)$
\item Given $v(u)$ and $u(g)$, the government chooses $g$ to maximize 
\begin{equation*}
\Uc(1 - [u(g) + v(u(g))] - g,g)
\end{equation*}
\end{itemize}
\end{frame}

\begin{frame}
\frametitle{Correcting the Samuelson formula}
\begin{itemize}
\item First-order condition of government's problem is
\begin{align*}
0 &=\pd{\Uc}{g} - \pd{\Uc}{c} - \pd{\Uc}{c}\cdot u'(g) \cdot\bs{1 + v'(u)}\\
1 &=\frac{\pdx{\Uc}{g}}{\pdx{\Uc}{c}} - u'(g) \cdot\bs{1 + v'(u)}
\end{align*}
\item Optimal public expenditure satisfies
\begin{equation*}
\underbrace{1 = MRS_{gc}}_{\text{Samuelson formula}} + \quad \al{\underbrace{[1 + v'(u)] \cdot [-u'(g)]}_{\text{correction}}}
\end{equation*}
\item $MRS_{gc} = [\pdx{\Uc}{g}]/[\pdx{\Uc}{c}]$: marginal rate of substitution between public and private consumption, decreasing in $g/c$
\item $[1 + v'(u)] \cdot [-u'(g)]$: correction to the Samuelson formula in presence of unemployment
\end{itemize}
\end{frame}

\begin{frame}
\frametitle{Interpretation of the corrected Samuelson formula}
\begin{equation*}
\underbrace{1 = MRS_{gc}}_{\text{Samuelson formula}} + \quad \underbrace{[1 + v'(u)] \cdot [-u'(g)]}_{\text{correction}}
\end{equation*}
\begin{itemize}
\item $MRS_{gc}$: $1$ when public goods $g$ and private goods $c$ are equally valuable, decreasing in $g/c$
\item $1 + v'(u)$: slope of $u+v(u)$, which is minimized at efficiency
\begin{itemize}
\item $1 + v'(u) < 0$ if the economy is inefficiently tight ($u<u^*$) 
\item $1 + v'(u) = 0$ if the economy is efficient ($u = u^*$) 
\item $1 + v'(u) > 0$ if the economy is inefficiently slack ($u>u^*$)
\end{itemize}
\item $-u'(g) = -du/dg = m$: unemployment multiplier, giving the reduction in \# unemployed workers with 1 extra public worker
\end{itemize}
\end{frame}

\begin{frame}
\frametitle{Departures from Samuelson rule}
\begin{table}
\begin{tabular*}{\textwidth}{@{\extracolsep\fill}lccc}
& \multicolumn{3}{c}{Multiplier} \\ 
\cmidrule{2-4}
State of economy & $-u'(g)<0$ & $-u'(g) = 0$ & $-u'(g) > 0$\\
\toprule
$1 + v'(u)>0$  & $\alr{MRS_{gc}>1}$  & $\alb{MRS_{gc}=1}$ & $\alg{MRS_{gc}<1}$ \\
$1 + v'(u)=0$ & $\alb{MRS_{gc}=1}$ & $\alb{MRS_{gc}=1}$  & $\alb{MRS_{gc}=1}$ \\ 
$1 + v'(u)<0$ & $\alg{MRS_{gc}<1}$ & $\alb{MRS_{gc}=1}$ & $\alr{MRS_{gc}>1}$ \\ 
\bottomrule
\end{tabular*}
\end{table}
\end{frame}

\begin{frame}
\frametitle{Departure of optimal spending $g/c$ from Samuelson spending $(g/c)^*$}
\begin{table}
\begin{tabular*}{\textwidth}{@{\extracolsep\fill}lccc}
& \multicolumn{3}{c}{Multiplier} \\ 
\cmidrule{2-4}
State of economy & $m<0$ & $m = 0$ & $m > 0$\\
\toprule
$u>u^{*}$  & $\alr{g/c<(g/c)^{*}}$  & $\alb{g/c=(g/c)^{*}}$ & $\alg{g/c>(g/c)^{*}}$ \\
$u=u^{*}$ & $\alb{g/c=(g/c)^{*}}$ & $\alb{g/c=(g/c)^{*}}$  & $\alb{g/c=(g/c)^{*}}$ \\ 
$u<u^{*}$ & $\alg{g/c>(g/c)^{*}}$ & $\alb{g/c=(g/c)^{*}}$ & $\alr{g/c<(g/c)^{*}}$ \\ 
\bottomrule
\end{tabular*}
\end{table}
\end{frame}

\begin{frame}
\frametitle{Interpretation of departure from Samuelson spending}
\begin{itemize}
\item Correction to the Samuelson formula appears due to effect of public expenditure on welfare through unemployment
\item Assume that public employment reduces unemployment ($ m> 0$) and the labor market is inefficiently slack ($u>u^*$)
\begin{itemize}
\item Then an increase in public employment shifts employment from the private to public sector (shift in the composition of the pie, as in Samuelson)
\item But it also increases the number of producers and therefore the total amount of production (increase in the size of the pie, absent from Samuelson)
\item This extra positive effect from public employment explains why the corrected formula recommends more public employment than Samuelson ($g/c > (g/c^*)$, or $MRS_gc<1$)
\end{itemize}
\end{itemize}	
\end{frame}

\begin{frame}
\frametitle{Explicit sufficient-statistic formula}
\begin{itemize}
\item Above formula only implicitly defines the optimal amount of public spending relative to private spending, $g/c$
\item Can rework the formula to express optimal $g/c$ as a function of fixed statistics:
\begin{equation*}
\frac{g/c-(g/c)^{*}}{(g/c)^{*}}\approx \frac{z_0\xi m}{1+  z_1 z_0 \xi m^{2}} \cdot \frac{u_{0}-u^{*}}{u^{*}}
\end{equation*}
\item Resulting unemployment $u-u^*$ is smaller than $u_0-u^*$ but positive:
\begin{equation*}
u-u^{*} \approx \frac{u_{0}-u^{*}}{1+ z_1 z_0 \xi m^{2}}>0
\end{equation*}
\item $u_0$: initial, inefficient unemployment rate
\item $\xi$: elasticity of substitution between public and private goods
\item $z_0$, $z_1$: constant of the parameters
\end{itemize}
\end{frame}


\begin{frame}
\frametitle{Illustration: US Great Recession \href{https://pascalmichaillat.org/6/}{(Michaillat, Saez 2019)} }
\begin{itemize}
\item Starting point: winter 2008--2009	
\item Unemployment $=6\%$ and public spending $=16.5\%$ of GDP
\begin{itemize}
	\item For illustration: we take these values as efficient so $u^* = 6\%$ and $(g/c)^* = 16.5\%$
\end{itemize}
\item Unemployment is forecast to increase to $9\%$
\begin{itemize}
	\item Initial unemployment gap $u_0 - u^*= 9\% - 6\% = 3\%$
\end{itemize}
\item We compute optimal stimulus for various unemployment multipliers $m$
\begin{itemize}
\item $\xi$, $z_0$, $z_1$: calibrated to US values
\end{itemize}
\item The resulting, optimal unemployment gap $u-u^*$ will be smaller than $u_0-u^*$ but positive
\end{itemize}
\end{frame}

\begin{frame}
\frametitle{Optimal stimulus spending (\% of GDP): small multiplier}
\includegraphics<1>[scale=\nfig,page=23]{\npdf}%
\end{frame}

\begin{frame}
\frametitle{Optimal stimulus spending (\% of GDP): medium multiplier}
\includegraphics<1>[scale=\nfig,page=24]{\npdf}%
\includegraphics<2>[scale=\nfig,page=25]{\npdf}%
\end{frame}

\begin{frame}
\frametitle{Optimal stimulus spending (\% of GDP): large multiplier}
\includegraphics[scale=\nfig,page=26]{\npdf}%
\end{frame}

\begin{frame}
\heading{Summary}
\end{frame}

\begin{frame}
\frametitle{Unemployment gap in the United States}
\begin{itemize}
\item Socially efficient unemployment rate $u^*$ \& unemployment gap $u-u^*$ are determined by 3 sufficient statistics
\begin{itemize}
\item Elasticity of Beveridge curve
\item Social cost of unemployment
\item Cost of recruiting
\end{itemize} 
\item In the United States, 1951--2019:
\begin{itemize}
\item $u^*$ averages $4.3\%$ \then $u-u^*$ averages $1.4$pp
\item $3.0\% < u^*  < 5.4\%$ \then $u-u^*$ is countercyclical
\item[\then] \al{Labor market is inefficient}
\item[\then] \al{Labor market is inefficiently slack in slumps}
\end{itemize}
\end{itemize}
\end{frame}


\begin{frame}
\frametitle{Implications for policy design}
\begin{itemize}
\item Optimal nominal interest rate is procyclical
\begin{itemize}
\item Optimal for monetary policy to eliminate the unemployment gap
\item Unemployment \down when interest rate \down
\end{itemize}
\item Optimal government spending is countercyclical
\begin{itemize}
\item Optimal for government spending to reduce---not eliminate---the unemployment gap
\item Unemployment \down when spending \up
\end{itemize}
\end{itemize}	
\end{frame}

\begin{frame}
\frametitle{Further implications for policy design}
\begin{itemize}
\item Optimal unemployment insurance is countercyclical \href{https://pascalmichaillat.org/4/}{(Landais, Michaillat, Saez 2018)}
\begin{itemize}
\item US tightness gap is procyclical	
\item Optimal for unemployment insurance to reduce---not eliminate---the tightness gap
\item Tightness \up when unemployment insurance \up
\end{itemize}
\item Optimal immigration policy is procyclical \href{https://pascalmichaillat.org/14/}{(Michaillat 2023)}
\begin{itemize}
\item Increase in immigration improves welfare when the labor market is inefficiently tight, and reduces welfare when labor market is inefficiently slack
\item Because immigration reduces labor market tightness
\end{itemize}
\end{itemize}	
\end{frame}


\end{document}
